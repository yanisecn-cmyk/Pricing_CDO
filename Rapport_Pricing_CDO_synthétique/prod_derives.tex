\section{Produits dérivés de crédit}
\subsection{Le marché du crédit risqué}

\subsubsection{Les obligations risquées}


\subsubsection{Spread de crédit}

\subsubsection{Notations de crédit}

\subsection{Credit Default Swap}
\subsubsection{Description du produit}

Un \emph{Credit Default Swap} (CDS) ou plus simplement \emph{swap} est un produit dérivé du crédit et peut être vu comme l'élément fondamental (ou sous-jacent) des produits plus exotiques comme les CDO synthétiques que nous verrons plus tard.

Sa fonction principale est de transférer le risque de crédit de référence d'une entreprise C (\emph{entité de référence}) entre deux contreparties A et B. Dans le contrat standard, l’une des parties en question, disons A, achète une protection contre le risque de perte en
cas de défaut de l’entité de référence C. Ce défaut est déclenché par un évènement de crédit formel spécifié dans le contrat. Cet événement peut être la faillite de l’entreprise, un défaut de paiement ou la restructuration de sa dette.

La protection est valable jusqu’à la maturité du swap. En échange de cette protection, l’acheteur A verse périodiquement (en général, tous les 3 mois) au vendeur B une prime et ce jusqu’au défaut de C ou jusqu’à maturité du swap. La jambe du swap correspondante est appelée \emph{premium leg}.

\begin{figure}[H]
    \centering
    \includegraphics[width=0.5\linewidth]{images/CDS-nodefault.png}
    \caption{Schéma de transaction d'un CDS sans défaut}
    \label{fig:CDS_no_default}
\end{figure}

Si le défaut intervient avant la maturité du swap, le vendeur de protection effectue un paiement à l’acheteur de protection. Ce paiement équivaut à la différence entre le nominal de la dette couverte par le swap et le taux de recouvrement observé à l’instant du défaut. Cette fois la jambe du swap correspondante est appelée \emph{protection leg}.

\begin{figure}[H]
    \centering
    \includegraphics[width=0.5\linewidth]{images/CDS-default.png}
    \caption{Schéma de transaction d'un CDS dans le cas d'un défaut}
    \label{fig:CDS_default}
\end{figure}

\subsubsection{Évaluation de la marge d’un CDS}

Considérons un CDS de maturité $T>0$ sur une entité de référence, portant sur un notionnel $N>0$. 
On note $(T_k)_{k=1,\dots,m}$ les dates de paiement de la jambe de prime (généralement trimestrielles), avec $0 < T_1 < \cdots < T_m = T$.  
La fraction d'année associée au coupon $k$ selon la convention du marché est notée $\delta_k = T_k-T_{k-1}$.
Considérons $\tau$ le temps de défaut de l'entité de référence, défini comme un temps d'arrêt, nous y associons un taux de recouvrement noté $R\in[0,1]$.

\paragraph{Jambe fixe (premium leg).}

À chaque date de coupon $T_k$, l'acheteur de protection paie un montant proportionnel au \emph{spread} (ou marge) $s$ (exprimé en taux annuel), au notionnel et à la fraction d'année. Ce paiement n’a lieu que si l’entité de référence n’a pas fait défaut avant $T_k$, c'est-à-dire si $\tau > T_k$. La valeur présente sous la mesure risque-neutre $\mathbb{P}^\ast$ de la jambe fixe est donc :
\begin{equation}
    \text{JF}(s)
= s\,N
\sum_{k=1}^m 
\delta_k \; \mathbb{E}^{\mathbb{P}^\ast}
\!\left[\frac{\mathbf{1}_{\{\tau > T_k\}}}{B_{T_k}}\right].
\end{equation}

\paragraph{Jambe variable (protection leg).}

En cas de défaut à un temps aléatoire $\tau \le T$, le vendeur de protection verse la \emph{loss given default} :
\begin{equation}
    \text{LGD} = N(1-R).
\end{equation}
La valeur présente de la jambe de protection ou jambe variable est alors
\begin{equation}
    \text{JV}
= N(1-R)\,
\mathbb{E}^{\mathbb{P}^\ast}
\!\left[\frac{\mathbf{1}_{\{\tau \le T\}}}{B_\tau}\right].
\end{equation}



\paragraph{Détermination du spread.}

Un CDS s’échange à valeur nulle à l’initiation. Le spread $s^\ast$ est donc défini par l’égalité :
\begin{equation}
    \text{JF}(s^\ast) = \text{JV}.
\end{equation}
On obtient :
\[
s^\ast
=
(1-R)\;
\frac{
\mathbb{E}^{\mathbb{P}^\ast}\!\left[\frac{\mathbf{1}_{\{\tau \le T\}}}{B_\tau}\right]
}{
\sum_{k=1}^m 
\delta_k \, \mathbb{E}^{\mathbb{P}^\ast}\!\left[\frac{\mathbf{1}_{\{\tau > T_k\}}}{B_{T_k}}\right]
}.
\]

\paragraph{Cas particulier : modèle à intensité déterministe.}

Dans un cadre standard où  le taux sans risque est constant : $r_t = r$ et l'intensité de défaut est une fonction déterministe : $\lambda : t \mapsto \lambda(t)$ la probabilité de survie est :
\begin{equation}
    S(t)=\exp\!\left(-\int_0^t \lambda(u)\, du\right),
\end{equation}
et la densité de défaut sous la mesure risque-neutre est $\lambda(t)\,S(t)$.

\noindent Alors la jambe variable est donnée par :

\begin{equation}
    \text{JV} = N(1-R) \int_0^T e^{-rt}\lambda(t)\,S(t)\,dt,
\end{equation}
et la jambe fixe par :
\begin{equation}
    \text{JF}(s) = s\,N \sum_{k=1}^m \delta_k \, e^{-rT_k} S(T_k).
\end{equation}
Le spread s'écrit :
\begin{equation}
    s^\ast
    =
    (1-R)\;
    \frac{
    \int_0^T e^{-rt}\lambda(t)\,S(t)\,dt
    }{
    \sum_{k=1}^m \delta_k \, e^{-rT_k} S(T_k)
    }.
\end{equation}

Ce spread constitue la prime d'assurance annuelle qui égalise la valeur actualisée des paiements fixes et celle du paiement contingent versé en cas de défaut de l'entité sous-jacente.



\subsection{Collateralized Debt Obligation}

Les \emph{Collateralized Debt Obligations} ou CDO sont des produits obligataires adossés
à des dettes, résultant d'un mécanisme relativement complexe d'ingénierie financière
appelé \emph{titrisation} (\emph{securitization}). A partir d'un panier de titres de dette (de 50 à
10000 créances), l'émetteur synthétise des actifs obligataires. Les CDO se distinguent
selon la nature de la dette sous-jacente : s'il s'agit de produits obligataires, on parle de
 “\emph{Collateralized Bond Obligations}” ou CBO. Dans le cas ou le panier est constitué
uniquement de titres de prêts, on parle de “Collateralized Loan Obligations” ou CLO.
Bien entendu, dans le cas général, le panier est mixte. Depuis sa création dans le milieu
des années 1990, le marché des CDO n'a cessé de se développer. En 2000, il dépassait
les 100 Milliards de dollars d'émission.

Après la crise des \emph{subprimes} de 2008 les CDO ont été pointés du doigt pour leur manque de
transparance et difficultées d'évaluation des risques. Suite à cela le marché a chuté
considérable\-ment cependant aujourd'hui ce marché représente toujours 23,4 milliards d'euros et
les projections actuelles lui prédisent près de 69 milliards d'euros en 2033.

Nous présentons dans cette partie les enjeux du processus de titrisation ainsi que ses mécanismes, puis les
techniques récentes liées à la génération synthétique de tranches utilisées en trading de corrélation.

\subsubsection{Titrisation}

La titrisation est une technique financière qui consiste à transférer à des investisseurs des
 actifs financiers tels que des créances (par exemple des factures émises non soldées, ou des
prêts en cours), en les transformant, par le passage à travers une structure \emph{ad hoc} --- souvent
un \emph{Special Purpose Vehicule}, une entité spécialement dédiée à absorbée les risques de ces
 produits --- en titres financiers émis sur le marché des capitaux.

\begin{figure}[H]
    \centering
    \includegraphics[width=0.5\linewidth]{images/Emission-CDO.png}
    \caption{Méchanisme d'émission d'un CDO}
    \label{fig:Emission-CDO}
\end{figure}

 L'intérêt est multiple, premièrement il permet à l'émetteur de crédit de transférer ces actifs et
 donc ces risques à une autre entité, il n'a donc pas besoin d'accumuler des liquidités pour couvrir
 les risques de défaut. Deuxièmement ces produits peuvent, comme nous le verrons ensuite, être les
 sous-jacents de produits dérivés. Enfin, le plus grand intérêt est le découpage des CDO en
\emph{tranches} de différents niveaux de risques à partir de sous-jacents de notations à priori
quelconques. Elles se décomposent sous cette forme :

\begin{itemize}
    \item la tranche \emph{junior} ou \emph{equity} supporte les premières pertes sur l'ensemble de
créances. Il s'agit donc d'un produit très risqué, payant un spread très élevé
à l'investisseur. Il s'agit d'un produit purement spéculatif ;
    \item la tranche intermédiaire, dite \emph{mezzanine} supporte les pertes au delà de la tranche
equity, c'est un produit moyennement risqué, offrant un spread intéressant ;
    \item la tranche \emph{senior} supporte les pertes restantes, si elles ont lieu. Elle est la moins
soumise au risque de crédit, et offre donc un coupon faible.
\end{itemize}

\begin{figure}[H]
    \centering
    \includegraphics[width=0.5\linewidth]{images/Waterfall.png}
    \caption{Distribution des tranches et effet "\emph{waterfall}" d'un CDO}
    \label{fig:Waterfall}
\end{figure}

\subsubsection{Les produits synthétiques}

Devant le développement impressionnant du marché des CDO et la demande de
produits de corrélation de plus en plus forte de la part des investisseurs,
des techniques d'ingénierie financière ont donné naissance au concept des CDO
synthétiques.

De l'utilisation originelle dans les stratégies de gestion de fonds propres,
les CDO deviennent peu à peu des produits d'investissement spéculatifs.

Le principe de la titrisation synthétique est de constituer des produits tranchés à
partir, non plus d'un ensemble de crédits ou créances, mais d'un ensemble de CDS.
Ceci revient à dire que l'organisme émetteur créé une exposition au risque de crédit
et de corrélation en prenant des positions sur un ensemble de CDS.

L'intérêt mais également le plus grand risque de ces produits est qu'il n'est plus adossé
à un crédit en tant que tel mais véritablement à un CDS, swaps qui peuvent être émis en
quantités illimités et souvent non rapportée sur un même crédit --- en 2006 selon le journaliste Gregory Zuckerman il
y avait 5 fois plus de CDO synthétiques que classiques adossés à des prêts \emph{subprime} sur le marché. De plus, là où l'intestisseur d'un \emph{cash} CDO
classique engage les fonds \emph{ex ante} c'est à dire avant le défaut éventuel en achetant la tranche,
l'investisseur d'un CDO synthétique engage les fonds \emph{ex post} et donc uniquement en cas de défaut.
Ainsi, sans vérification de sa solvabilité, celui-ci peut s'exposer à un remboursement très important
sur un grand nombre de swap pour lesquels il ne s'attendait pas à ce qu'il y ait défaut (typiquement les
tranches \emph{senior} d'un CDO synthétique à priori sans risque).

En effet, dans le cas classique les sous-jacents constituant un CDO sont décoréllés géographi\-quement, typiquement
les crédits et les CDS associés à ceux-ci proviennent de zones géographiques diverses. La probabilité que tous ces
swaps s'activent au sein du CDO est donc infime en théorie et donc les tranches supérieures ne devraient jamais
être impactées même pour des crédits de très mauvaise qualité.

\subsubsection{Un exemple de CDO synthétique}