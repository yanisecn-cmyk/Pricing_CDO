\section{Pricing des tranches de CDO synthétiques}

\subsection{Structure et définition mathématique des tranches}

Un CDO synthétique est structuré sur un portefeuille de référence composé de $n$
entités, chacune associée à un CDS. Notons $N_i$ le notionnel de l'entité $i$, $R_i$
son taux de recouvrement, et $\tau_i$ son temps de défaut. Le notionnel total du
portefeuille s'écrit $N_{\text{tot}} = \sum_{i=1}^n N_i$.
La perte cumulée du portefeuille à l'instant $t$ est définie par
\begin{equation}
L(t) = \sum_{i=1}^n N_i(1-R_i) \mathbb{1}_{\{\tau_i \leq t\}}.
\end{equation}
Cette quantité représente la somme des pertes effectives dues aux défauts survenus
avant ou à l'instant $t$. Il est commode de normaliser cette perte en proportion
du notionnel total et d'introduire le taux de perte du portefeuille
\begin{equation}
\ell(t) = \frac{L(t)}{N_{\text{tot}}} = \frac{1}{N_{\text{tot}}} \sum_{i=1}^n N_i(1-R_i) \mathbb{1}_{\{\tau_i \leq t\}}.
\end{equation}
Le capital du CDO est divisé en tranches caractérisées par leurs points d'attachement
et de détachement.
Une tranche $[K_1, K_2]$ est définie par un point d'attachement inférieur
$K_1$ et un point de détachement supérieur $K_2$, avec $0 \leq K_1 < K_2 \leq 1$. 
La tranche absorbe les pertes du portefeuille comprises entre $K_1 \cdot N_{\text{tot}}$ 
et $K_2 \cdot N_{\text{tot}}$. La perte de la tranche à l'instant $t$ s'exprime comme
\begin{equation}
L_{\text{tranche}}(t) = \min\left(\max(L(t) - K_1 N_{\text{tot}}, 0), (K_2 - K_1) N_{\text{tot}}\right),
\end{equation}
ou de manière équivalente en termes normalisés
\begin{equation}
\ell_{\text{tranche}}(t) = \min(\max(\ell(t) - K_1, 0), K_2 - K_1).
\end{equation}
La tranche subit des pertes uniquement lorsque le taux de perte du portefeuille
dépasse son point d'attachement inférieur $K_1$, et ces pertes sont plafonnées à
l'épaisseur de la tranche $K_2 - K_1$. En pratique, les CDO synthétiques sont structurés 
en plusieurs tranches de subordination différente. Par exemple, une distribution de tranches classiques
pourrait prendre la forme suivante : Equity $[0\%, 5\%]$, Mezzanine $[5\%, 12\%]$,
Senior $[12\%, 15\%]$, et Super-Senior $[15\%, 100\%]$.

\subsection{Perte espérée et distribution des pertes}

L'évaluation d'une tranche de CDO repose sur la connaissance de la distribution de probabilité
de sa perte $L_{\text{tranche}}(t)$ sous la mesure risque-neutre $\mathbb{P}^*$.
Une quantité centrale est la perte espérée de la tranche à la maturité $T$, définie par
\begin{equation}
\text{EL}_{\text{tranche}} = \mathbb{E}^{\mathbb{P}^*}[L_{\text{tranche}}(T)] = \mathbb{E}^{\mathbb{P}^*}\left[\min\left(\max(L(T) - K_1 N_{\text{tot}}, 0), (K_2 - K_1) N_{\text{tot}}\right)\right].
\end{equation}

Le calcul de cette espérance nécessite la connaissance de la distribution jointe des temps
de défaut $(\tau_1, \ldots, \tau_n)$, obtenue par les modèles de copule ou d'intensité 
stochastique présentés précédemment. 

Pour calculer la perte espérée, il est utile d'introduire la fonction de répartition de la perte
du portefeuille. Notons $F_L(x, t) = \mathbb{P}^*(L(t) \leq x)$ la probabilité que la perte cumulée
à l'instant $t$ n'excède pas $x$. La perte espérée de la tranche peut alors s'exprimer comme une
intégrale faisant intervenir cette distribution. En effet, en notant $\ell = L/N_{\text{tot}}$ le taux
de perte normalisé, nous avons
\begin{equation}
\text{EL}_{\text{tranche}} = N_{\text{tot}} \int_{K_1}^{K_2} \mathbb{P}^*(\ell(T) > x) \, dx.
\end{equation}
La distribution $F_L(x,t)$ n'admet généralement pas de forme analytique, sauf dans des cas très particuliers.
Le calcul numérique de cette distribution et de l'expected loss repose sur les méthodes de simulation 
de Monte Carlo.

\subsection{Valorisation des tranches et calcul du spread}

Une tranche de CDO synthétique peut être vue comme un CDS dont le sous-jacent est le portefeuille
de référence, mais avec une subordination définie par les points d'attachement et de détachement.
L'acheteur de protection sur une tranche paie un spread périodique $s$ et reçoit une compensation en cas
de pertes affectant la tranche. Le vendeur de protection perçoit le spread et s'engage à compenser les pertes de la tranche.

La valorisation de la tranche repose sur l'égalité entre la valeur actualisée de la jambe premium
ou jambe fixe (payée par l'acheteur de protection) et la valeur actualisée de la jambe protection 
ou jambe variable (payée par le vendeur).
Cette condition d'absence d'arbitrage permet de déterminer le spread équitable de la tranche.

\subsubsection{Jambe premium}

L'acheteur de protection paie le spread $s$ de manière périodique aux dates $t_1, t_2, \ldots, t_m = T$,
avec une fréquence généralement trimestrielle. Le paiement à chaque date est proportionnel au notionnel
restant de la tranche, c'est-à-dire à la partie de la tranche qui n'a pas encore subi de pertes. Le notionnel
restant à l'instant $t_i$ est
\begin{equation}
N_{\text{restant}}(t_i) = (K_2 - K_1) N_{\text{tot}} - L_{\text{tranche}}(t_i).
\end{equation}
La valeur actualisée de la jambe premium s'écrit
\begin{equation}
\text{JF}(s) = s \sum_{i=1}^m \delta_i    \mathbb{E}^{\mathbb{P}^\ast} \left[ \frac{N_{\text{restant}}(t_i)}{B_{t_i}} \right],
\end{equation}
où $\delta_i = t_i - t_{i-1}$ est la fraction d'année entre deux paiements
 et $B_{t_i}$ est le numéraire.

\subsubsection{Jambe protection}

Le vendeur de protection s'engage à compenser les pertes subies par la tranche. Lorsqu'un défaut
survient dans le portefeuille et que la perte cumulée du portefeuille atteint ou dépasse le point 
d'attachement de la tranche, le vendeur effectue un paiement correspondant à l'augmentation de la 
perte de la tranche. La valeur actualisée de la jambe protection s'exprime comme
\begin{equation}
\text{JV} = \mathbb{E}^{\mathbb{P}^\ast}\left[\int_0^T \frac{dL_{\text{tranche}}(t)}{B_t} \, \right],
\end{equation}
où $dL_{\text{tranche}}(t)$ représente l'incrément de perte de la tranche à l'instant $t$. En pratique, cette intégrale est approchée par une somme discrète sur les dates d'observation :
\begin{equation}
\text{JV} \approx \sum_{i=1}^m  \mathbb{E}^{\mathbb{P}^\ast}\left[ \frac{L_{\text{tranche}}(t_i) - L_{\text{tranche}}(t_{i-1})}{B_{t_i}} \right].
\end{equation}

\subsubsection{Spread équitable}

Le spread équitable $s^*$ de la tranche est déterminé par la condition d'absence d'arbitrage imposant l'égalité entre les deux jambes :
\begin{equation}
\text{PF}(s^*) = \text{JV}.
\end{equation}
En explicitant cette condition et en isolant $s^*$, nous obtenons
\begin{equation}
s^* \approx \frac{\sum_{i=1}^m  \mathbb{E}^{\mathbb{P}^\ast}[B_{t_i}^{-1} (L_{\text{tranche}}(t_i) - L_{\text{tranche}}(t_{i-1}))]}{\sum_{i=1}^m \delta_i  \mathbb{E}^{\mathbb{P}^\ast}[B_{t_i}^{-1}] [(K_2 - K_1) N_{\text{tot}} - \mathbb{E}^{\mathbb{P}^\ast}[L_{\text{tranche}}(t_i)]]}.
\end{equation}