\section{Méthodes numériques}

\subsection{Simulation des taux d'intérêts}


\subsection{Calibration des copules}

Nous cherchons à calibrer une copule gaussienne à partir des spreads
de CDS observés sur les 10 dates fournies. Pour cela nous allons 
chercher à minimiser la norme euclidienne entre les spreads observés et les 
spreads calculés à partir de la copule gaussienne. Nous allons utiliser
la fonction d'optimisation de Scipy \texttt{scipy.optimize.minimize} pour 
trouver les paramètres de la copule gaussienne qui minimisent cette norme.

Afin de simplifier le problème d'optimisation, dans une première étape nous
allons supposer que les corrélations entre les différents CDS sont toutes 
égales, c'est à dire que la matrice de corrélation est de la forme
\begin{equation}
\Sigma = \begin{pmatrix}1 & \rho & \rho & \dots & \rho \\
\rho & 1 & \rho & \dots & \rho \\
\rho & \rho & 1 & \dots & \rho \\
\vdots & \vdots & \vdots & \ddots & \vdots \\
\rho & \rho & \rho & \dots & 1 \\
\end{pmatrix}
\end{equation}
où $\rho$ est le paramètre de corrélation implicite que nous allons calibrer. Nous
sommes donc dans le cas de l'approximation d'un portfolio homogène.

Pour le trouver nous procédons en deux temps, premièrement nous utilisons une
grille de paramètres sur laquelle nous calculons les spreads pour chaque tranche
par une méthode de Monte-Carlo en échantillonnant des temps de défauts du vecteur
aléatoire $(\tau_1, \dots, \tau_n)$. Puis sur une région plus restreinte nous
utilisons l'algorithme de Broyden-Fletcher-Goldfarb-Shanno (BFGS) pour trouver
le minimum de la norme euclidienne entre les spreads observés et les spreads calculés 
à partir de la copule gaussienne. Il s'agit d'une méthode approchée de la méthode
de Newton dite de "quasi-Newton".

Dans une recherche ultérieure nous pourrons appliquer ces méthodes à des problèmes
d'optimisation en plus grande dimension en essayant de calibrer une matrice de corrélation
non homogène ou bien d'autres types de copules comme celle de Student ou bien d'autres
mixtures.








