\section{Méthodes numériques}

\subsection{Simulation des taux d'intérêts}

\paragraph{Principe général de la simulation et différence entre les modèles.}

\vspace{0.5cm}

Dans les deux cas, les formules utilisées pour simuler $r_{t+\Delta t}$ correspondent à l’échantillonnage de la loi de transition exacte du taux court, et non à un schéma de discrétisation de type Euler.

Dans le modèle de Vasicek, l’équation étant linéaire, le processus est gaussien et la loi conditionnelle de $r_{t+\Delta t}$ sachant $r_t$ est normale, ce qui conduit à une formule de simulation explicite.  
Dans le modèle CIR, le coefficient de diffusion dépend de $\sqrt{r_t}$, l’équation n’est plus linéaire et la loi conditionnelle de transition est une loi du $\chi^2$ non centrale.

La différence entre les expressions de simulation provient donc directement de la structure des équations différentielles stochastiques des deux modèles.

\subsubsection{Simulation du modèle de Vasicek}

Le modèle de Vasicek est linéaire et admet une solution explicite sous forme d'intégrale stochastique :

\begin{equation}
    r_{t+\Delta t} = r_t e^{-a \Delta t} + b (1 - e^{-a \Delta t}) + \sigma \sqrt{\frac{1 - e^{-2 a \Delta t}}{2a}} \, Z_t,
\end{equation}

où $Z_t \sim \mathcal{N}(0,1)$ est une variable aléatoire normale standard indépendante.

\textbf{Méthode de simulation :}
\begin{enumerate}
    \item Choisir un pas de temps $\Delta t$ et un horizon final $T$.
    \item Initialiser $r_0$.
    \item Générer des variables normales $Z_0, Z_1, \dots$.
    \item Calculer itérativement $r_{t+\Delta t}$ à partir de $r_t$ en utilisant la formule ci-dessus.
\end{enumerate}


\medskip
\paragraph{Calcul du numéraire dans le modèle de Vasicek}

On rappelle que le numéraire associé au marché monétaire est défini par
\begin{equation}
    B_t=\exp\left(\int_0^t r_s\,ds\right).
\end{equation}

Dans le cadre du modèle de Vasicek, le processus $(r_t)_{t\geq 0}$ est un processus gaussien à trajectoires continues et vérifie la propriété d’intégrabilité
\[
\int_0^T |r_s|\,ds<\infty \quad \text{p.s.}
\]
comme établi dans la section précédente.

Par conséquent, pour toute trajectoire simulée $(r_{t_k})_{k=0,\ldots,N}$, une approximation naturelle du numéraire s’obtient par une discrétisation de l’intégrale :
\begin{equation}
    B_{t_{k+1}}
    =
    B_{t_k}\exp\!\left(r_{t_k}\Delta t\right),
    \qquad B_0=1,
\end{equation}
où $\Delta t=t_{k+1}-t_k$.

Cette formule découle directement de la définition intégrale de $B_t$ et du fait que, pour des trajectoires continues, la somme de Riemann
\[
\sum_{k=0}^{N-1} r_{t_k}\Delta t
\]
converge presque sûrement vers $\int_0^T r_s\,ds$ lorsque $\Delta t\to 0$.

Ainsi, la simulation exacte du taux court $r_t$ dans le modèle de Vasicek permet de calculer de manière cohérente le numéraire par simple intégration numérique le long des trajectoires simulées.



\subsubsection{Simulation du modèle CIR}

Le modèle CIR ne possède pas de solution intégrale simple, mais il est possible de simuler $r_t$ de manière exacte grâce à la relation avec la loi du $\chi^2$ non centrale.

\textbf{Méthode de simulation :}
\begin{enumerate}
    \item Choisir un pas de temps $\Delta t$ et un horizon $T$.
    \item Initialiser $r_0 > 0$.
    \item À chaque pas, générer $r_{t+\Delta t}$ selon :
    \[
        r_{t+\Delta t} = \frac{\sigma^2 (1 - e^{-a \Delta t})}{4a} \, \chi^2_{\nu}(\lambda),
    \]
    où $\nu = \frac{4ab}{\sigma^2}$ et $\lambda = \frac{4 a r_t e^{-a \Delta t}}{\sigma^2 (1 - e^{-a \Delta t})}$.
\end{enumerate}


\begin{itemize}
    \item La simulation des trajectoires du modèle de Vasicek est simple et rapide grâce à la solution analytique.
    \item Le modèle CIR garantit la positivité du taux, mais nécessite la génération de variables $\chi^2$ non centrales.
\end{itemize}

\medskip
\paragraph{Calcul du numéraire dans le modèle CIR}

Dans le modèle de Cox--Ingersoll--Ross, le numéraire est également défini par
\begin{equation}
    B_t=\exp\left(\int_0^t r_s\,ds\right).
\end{equation}

Sous les hypothèses du modèle, et en particulier sous la condition de Feller $2ab\geq \sigma^2$, le processus $r_t$ est strictement positif et à trajectoires continues.

On a donc, pour tout horizon fini $T$,
\[
\int_0^T r_s\,ds<\infty \quad \text{p.s.}
\]
ce qui garantit que le numéraire est bien défini et strictement positif.

Comme dans le cas de Vasicek, le calcul du numéraire repose sur une approximation de l’intégrale par une discrétisation temporelle. Pour une grille régulière $(t_k)$, on pose
\begin{equation}
    B_{t_{k+1}}
    =
    B_{t_k}\exp\!\left(r_{t_k}\Delta t\right),
    \qquad B_0=1.
\end{equation}

Cette approximation est justifiée par la continuité presque sûre de $r_t$, qui assure la convergence de la somme de Riemann vers l’intégrale stochastique déterministe
\[
\int_0^t r_s\,ds.
\]

Le fait que la simulation du processus CIR soit effectuée à l’aide de sa loi exacte à chaque pas de temps ne modifie pas ce calcul : l’intégrale du taux court ne possède pas de loi fermée simple et doit être évaluée numériquement le long des trajectoires simulées.




\subsection{Calibration des copules}

Nous cherchons à calibrer une copule gaussienne à partir des spreads
de CDS observés sur les 10 dates fournies. Pour cela nous allons 
chercher à minimiser la norme euclidienne entre les spreads observés et les 
spreads calculés à partir de la copule gaussienne. Nous allons utiliser
la fonction d'optimisation de Scipy \texttt{scipy.optimize.minimize} pour 
trouver les paramètres de la copule gaussienne qui minimisent cette norme.

Afin de simplifier le problème d'optimisation, dans une première étape nous
allons supposer que les corrélations entre les différents CDS sont toutes 
égales, c'est à dire que la matrice de corrélation est de la forme
\begin{equation}
\Sigma = \begin{pmatrix}1 & \rho & \rho & \dots & \rho \\
\rho & 1 & \rho & \dots & \rho \\
\rho & \rho & 1 & \dots & \rho \\
\vdots & \vdots & \vdots & \ddots & \vdots \\
\rho & \rho & \rho & \dots & 1 \\
\end{pmatrix}
\end{equation}
où $\rho$ est le paramètre de corrélation implicite que nous allons calibrer. Nous
sommes donc dans le cas de l'approximation d'un portfolio homogène.

Pour le trouver nous procédons en deux temps, premièrement nous utilisons une
grille de paramètres sur laquelle nous calculons les spreads pour chaque tranche
par une méthode de Monte-Carlo en échantillonnant des temps de défauts du vecteur
aléatoire $(\tau_1, \dots, \tau_n)$. Puis sur une région plus restreinte nous
utilisons l'algorithme de Broyden-Fletcher-Goldfarb-Shanno (BFGS) pour trouver
le minimum de la norme euclidienne entre les spreads observés et les spreads calculés 
à partir de la copule gaussienne. Il s'agit d'une méthode approchée de la méthode
de Newton dite de "quasi-Newton".

Dans une recherche ultérieure nous pourrons appliquer ces méthodes à des problèmes
d'optimisation en plus grande dimension en essayant de calibrer une matrice de corrélation
non homogène ou bien d'autres types de copules comme celle de Student ou bien d'autres
mixtures.








