\section{Introduction}
\subsection{Motivation}

Depuis les années 1990, le développement des marchés de crédit a donné naissance à une classe de produits dérivés complexes destinés à transférer, mutualiser et redistribuer le risque de défaut : les \textit{Collateralized Debt Obligations} (CDO) ou titre de créance collatéralisé en français. Initialement introduits par des institutions financières telles que \textit{Drexel Burnham Lambert} à la fin des années 1980, puis massivement développés par \textit{J.P. Morgan} au cours de la décennie suivante, les CDO avaient pour ambition d’optimiser l’allocation du risque de crédit en permettant la titrisation de portefeuilles d’actifs hétérogènes. Ils offraient aux investisseurs la possibilité de prendre des expositions ajustées au risque grâce à une structure hiérarchisée en tranches (\textit{equity}, \textit{mezzanine} et \textit{senior}), chacune absorbant une fraction distincte des pertes éventuelles.

Les \textit{CDO synthétiques}, reposant non pas sur des obligations physiques mais sur des contrats de \textit{Credit Default Swap} (CDS), ont marqué une étape importante dans cette évolution. Présentés comme plus flexibles, plus liquides et plus rapides à structurer, ils permettaient aux institutions financières d’accroître ou de couvrir leurs expositions sur des portefeuilles de crédit sans détenir directement les actifs sous-jacents. Cette innovation a contribué à l’expansion rapide du marché des produits structurés au cours des années 2000.

Cependant, la crise financière de 2007--2008 a mis en lumière les risques systémiques liés à ces instruments. Leur complexité intrinsèque, la difficulté d’estimer correctement les corrélations de défaut et les limites du modèle de copule gaussienne largement utilisé à l’époque ont conduit à une sous-estimation significative des risques réels associés à certaines tranches, en particulier les tranches \textit{mezzanine} et \textit{senior}. Ces insuffisances de modélisation et de calibration ont joué un rôle non négligeable dans l’amplification de la crise.

Dans ce contexte, une compréhension rigoureuse des mécanismes de valorisation des CDO synthétiques, notamment des modèles de dépendance et des dynamiques de défaut, demeure essentielle. La capacité à tarifier correctement ces instruments est déterminante pour la gestion du risque et la stabilité financière. Le présent rapport s’inscrit dans cette perspective : il vise à étudier, formaliser et comparer plusieurs approches de modélisation du risque de défaut et de tarification des tranches synthétiques, en particulier les modèles à copules et les modèles à intensité.

\subsection{Cadre probabiliste et hypothèses sur le taux instantané}

\subsubsection{Espace probabilisé et filtration}

Nous nous plaçons dans un cadre continu en temps sur un horizon fini $[0,T]$, muni
d'un espace probabilisé complet
\[
(\Omega,\mathcal{F},\mathbb{P},(\mathcal{F}_t)_{t\ge 0},),
\]
où la filtration $(\mathcal{F}_t)$ satisfait les conditions usuelles:

\begin{itemize}
    \item Continuité à droite : $\mathcal{F}_t = \bigcap_{s>t} \mathcal{F}_s$
    \item Complétude : Soit $N$ un ensemble négligeable, alors $N \subset \mathcal{F}_0$
\end{itemize}

\noindent Tous les processus considérés sont supposés adaptés à $(\mathcal{F}_t)$ c'est à dire, pour tout processus $(X_t)_{t\ge0}$, pour tout $t\ge0$, la variable aléatoire $X_t$ est $\mathcal{F}_t$-mesurable.

\subsubsection{Intérêts composés, numéraire et taux instantanés}

\paragraph{Cas déterministe}

Considérons un taux annuel constant $r \in \mathbb{R}$, un capital initial $N_0 = 1$ et
cherchons à calculer le capital au temps $T$ avec $n$ compositions par an :
\begin{equation}
    N_T^{(n)} = \left ( 1 + \frac{r}{n} \right )^{nT}
\end{equation}
Or :
\begin{equation}
    \lim_{n \to +\infty} \left ( 1 + \frac{r}{n} \right )^{nT} = e^{rT}
\end{equation}
Donc le capital en capitalisation continue est :
\begin{equation}
    N_T = e^{rT}
\end{equation}
A présent découpons l'intervalle $[0, T]$ en $n$ sous-intervalles :
\begin{equation}
    0 = t_0 < t_1 < ... < t_n = T, \quad \Delta t = \frac{T}{n}
\end{equation}
Et supposons sur chaque intervalle $[t_i, t_{i+1}[$ le taux constant et égal à $r(t_i)$.
Nous pouvons alors définir le schéma d'Euler explicite tel que :
\begin{equation}
    \begin{cases}
        N_0^{(n)} = 1 \\
        N_{t_{i+1}}^{(n)} = N_{t_{i}}^{(n)} \left (1 + r(t_{i}) \Delta t \right )
    \end{cases}
\end{equation}
solution de l'équation différentielle ordinaire:
\begin{equation}
    \dot{N}_t = r(t) N_t
\end{equation}
Sous l'hypothèse d'intégrabilité de $r$ sur $[0, T]$, par le théorème de Cauchy-Lipschitz 
\footnote{En posant $f(t,N)=r(t)N$, $r$ est localement bornée sur $[0, T]$ donc $f$ est
localement lipschitzienne et continue presque partout par rapport à la deuxième variable ce qui vérifie les
hypothèses du théorème.} il existe une unique solution à
ce problème que l'on nommera \emph{numéraire} : un actif sans risque $N=(N_t)_{t\ge 0}$
vérifiant
\begin{equation}
    N_t = \exp\!\left(\int_0^t r_s \, ds\right),
    \label{rates_formula}
\end{equation}
Ce numéraire nous offre une mesure de la valeur monétaire au temps $t$,
pour 1€ placé en banque au temps initial on récupère $N_t$ au temps $t$.

\paragraph{Cas stochastique}

Cette définition s'étend naturellement dans le cas où $r_t$ suit le processus stochastique
de \emph{taux instantané} (ou \emph{taux court}) $(r_t)_{t \ge 0}$ vérifiant les hypothèses suivantes :

\begin{itemize}
    \item \textbf{(H1) Adaptation et mesurabilité :}
    $r_t$ est $(\mathcal{F}_t)$-progressivement mesurable.
    %
    \item \textbf{(H2) Bornes ou conditions d'intégrabilité :}
    $\displaystyle \int_0^T |r_s|\,ds < \infty$ p.s., ce qui garantit que 
    $B_t>0$ est bien défini et continu.
    %
    \item \textbf{(H3) Modélisation stochastique :}
    $r_t$ est généralement supposé être une diffusion de la forme :
    \begin{equation}
        dr_t = \mu(t,r_t)\,dt + \sigma(t,r_t)\,dW_t,
    \end{equation}
    où $(W_t)$ est un Brownien sous $\mathbb{P}$, et les coefficients
    $\mu$, $\sigma$ satisfont des conditions de Lipschitz et croissance linéaire
    assurant l'existence et l'unicité forte de la solution.
\end{itemize}

Cette fois, nous définissons l'équation différentielle stochatique suivante :
\begin{equation}
    \begin{cases}
        N_0 = 1 \\
        dN_t = r_t N_t dt
    \end{cases}
\end{equation}
Cette EDS est en réalité une équation différentielle aléatoire puisqu'elle
ne comporte pas de terme de diffusion mais ses coefficients sont aléatoires. Nous cherchons
ainsi à déterminer le champ mesurable 
\begin{equation}
    N : [0, T] \time \Omega \to \mathbb{R}
\end{equation}
vérifiant, pour tout $t \in [0, T]$,
\begin{equation}
    N_t(\omega) = 1 + \int_0^t r_s(\omega)B_s(\omega) ds, \quad \text{p.s.}
\end{equation}
Remarquons que pour $\omega \in \Omega$ fixé cette égalité est
déterministe, l'intégrale est de Lebesgue classique.

Cette fois, en posant $f(t, x, \omega) = r_t(\omega) x$ nous vérifions
que $f$ est continue et localement lipschitzienne en $x$ et intégrable en $t$,
il existe donc une unique solution chemin par chemin à notre équation différentielle par le
théorème de Carathéodory \footnote{Dans le cas stochastique $(r_t)_{t\ge0}$ suit un processus non continu de manière
générale mais mesurable donc le théorème de Carathéodory s'applique mais pas Cauchy-Lipschitz.} pour les EDO. 
En séparant les variables et en résolvant pour tout $\omega \in \Omega$ vérifiant $\mathbf{(H2)}$
nous obtenons la solution suivante :
\begin{equation}
    N_t = \exp\!\left(\int_0^t r_s \, ds\right), \quad \text{p.s.}
\end{equation}



\subsubsection{Processus de prix des actifs risqués}

\begin{definition}[Martingale locale]
Soit $(\Omega,\mathcal F,(\mathcal F_t)_{t \ge 0},\mathbb P)$ un espace probabilisé filtré satisfaisant les
conditions usuelles. Un processus adapté \emph{càdlàg} \footnote{Un processus \emph{càdlàg} est un processus
dont les trajectoires sont presque sûrement continues à droite et dont des limites à gauche existent partout.}
$(M_t)_{t \ge 0}$ est appelé \emph{martingale locale} s'il existe une suite croissante de temps d'arrêt
$(\tau_n)_{n \in \mathbb N}$ telle que
\[
\tau_n \to +\infty \quad \text{p.s.},
\]
et que, pour tout $n$, le processus arrêté
\[
(M_{\min(t, \tau_n)})_{t \ge 0}
\]
est une martingale (intégrable) par rapport à $(\mathcal F_t)$.
\end{definition}

Soit $S=(S_t)_{t \ge 0}$ un actif risqué. Sous la probabilité historique $\mathbb{P}$, sa dynamique est
supposée être une \emph{semi-martingale}, c'est-à-dire un processus pouvant être décomposé
en une \emph{locale martingale} et un processus adapté càdlag à variations locales finies, typiquement
un processus de diffusion :
\begin{equation}
    dS_t = S_t\big( b_t\,dt + \sigma_t\,dW_t \big),
\end{equation}
où $(b_t)$ est la dérive ou ''\textit{drift}'' (prime de risque incluse) et $(\sigma_t)$ la volatilité.

\noindent On note les prix actualisés par le numéraire :
\begin{equation}
    \tilde S_t := \frac{S_t}{B_t}.
\end{equation}

\subsubsection{Hypothèse d'absence d'arbitrage et existence de la mesure risque-neutre}

\begin{definition}[Mesures équivalentes]
    Soient $\mu$ et $\nu$ deux mesures sur l'espace mesurable $(X,\mathcal{F})$ et notons
    \[
        \mathcal{N}_\mu := \{ A \in \mathcal{F} \mid \mu(A) = 0 \}, 
        \quad \mathcal{N}_\nu := \{ A \in \mathcal{F} \mid \nu(A) = 0 \}
    \]
    les mesures $\mu$ et $\nu$ sont dites \emph{équivalentes} si $\mathcal{N}_\mu = \mathcal{N}_\nu$
    et nous noterons $\mu \sim \nu$.
\end{definition}

Selon le théorème fondamental de l'évaluation des actifs \cite{DelbaenSchachermayer},
l'hypothèse \emph{Absence of free lunch with vanishing risk} \footnote{Cette hypothèse est complexe à définir
mathématiquement et ne nous apporte rien d'intéressant, en revanche c'est celle-ci qui justifie l'existence
de la mesure équivalente risque-neutre. Nous pouvons l'interpréter comme une absence de stratégie ayant un gain
sans risque réel ou investissement initial.} (NFLVR) est équivalente à l'existence d'une probabilité
$\mathbb{P}^* \sim \mathbb{P}$ (appelée \emph{mesure équivalente martingale} ou \emph{mesure risque-neutre})
telle que les prix actualisés soient des \emph{martingales locales} sous
$\mathbb{P}^*$ :
\[
\tilde S_t = \frac{S_t}{N_t}
\quad\text{est une }\mathbb{P}^*\text{-locale martingale}.
\]
Sous cette mesure, la dynamique de $S_t$ s'écrit :
\begin{equation}
    dS_t = S_t\big( r_t\,dt + \sigma_t\,dW_t^* \big),
\end{equation}
où $(W_t^*)$ est un processus de Wiener sous $\mathbb{P}^*$.

\subsection{Le modèle de crédit}

Nous cherchons à présent à connaître le prix juste $\Pi_X(t,T)$ (i.e. sans prime de risque) à payer
en $t$ pour acheter un actif ayant une revendication ou \emph{payoff} $X\in L^1(\mathcal{F}_T,\mathbb{P}^*)$
en $T$. 

Nous commençons donc par actualiser la valeur de $X$ en $T$ par le numéraire $B_T$ (car 1€ demain vaut moins que
1€ aujourd'hui) et nous estimons cette quantité inconnue en prenant son espérance sous un univers sans prime de
risque conditionnellement à l'information disponible en $t$ c'est à dire $\mathcal{F}_t$. En normalisant par le
numéraire à l'instant présent nous obtenons :
\begin{equation}
    \Pi_X(t,T) = N_t\, \mathbb{E}^{\mathbb{P}^*}\!\left[ \frac{X}{N_T} \,\middle|\, \mathcal{F}_t \right].
\end{equation}
Cette formule suit directement de la propriété de martingale de $(\tilde S_t)$.

A partir de cette formule il est possible de définir l'obligation zéro-coupon (i.e. à prix juste) sans risque,
il s'agit d'un actif ayant une revendication de 1€ au temps $T$ dont la valeur est donnée par
\begin{equation}
    B(t,T) = N_t\, \mathbb{E}^{\mathbb{P}^*}\!\left[ \frac{1}{N_T} \,\middle|\, \mathcal{F}_t \right]
\end{equation}
Il est alors possible de rentrer le terme $N_t$ dans l'espérance puisque celui-ci est constant conditionnellement
à l'information $\mathcal{F}_t$ et en remplaçant par l'expression du numéraire nous obtenons :
\begin{equation}
    B(t,T) = \mathbb{E}^{\mathbb{P}^*}\!\left[ \exp\!\left(-\int_t^T r_s \, ds\right) \,\middle|\, \mathcal{F}_t \right]
\end{equation}
Si le risque de contrepartie de l'émetteur du zéro-coupon n'est pas nul, l'évaluation du
zéro-coupon doit tenir compte de la possibilité du défaut de celui-ci : deux nouveaux
risques entrent en jeu :
\begin{itemize}
    \item l'instant de défaut,
    \item la perte en cas de défaut (\emph{Loss Given Default}).
\end{itemize}
La perte en cas de défaut s'exprime comme un taux de recouvrement $R$ éventuellement aléatoire et d'une hypothèse de
recouvrement. Nous notons $D(t,T)$ la valeur en $t$ d'un zéro-coupon risqué de maturité $T$ et $\tau$ l'instant de
défaut de l'émetteur de ce titre. Il existe plusieurs hypothèses de recouvrement en théorie des mathématiques financières
mais nous utiliserons la plus courante dite \emph{recovery of par value} et qui consiste en le recouvrement
à l'instant de défaut d'une fraction $R$ du nominal du titre
\begin{equation}
    D(t,T) = \mathbb{E}^{\mathbb{P}^*}\!\left[ e^{- \int_t^T r_s \, ds} \mathbb{1}_{\{ \tau > T \}} + Re^{- \int_t^T r_s \, ds} \mathbb{1}_{\{ t < \tau < T \}}\,\middle|\, \mathcal{F}_t \right]
\end{equation}
Dans la suite du document nous chercherons à évaluer le prix des actifs à l'instant initial $t_0 = 0$, les prix
seront donc donnés par des espérances conditionnellement à l'information initiale $\mathcal{F}_0$. Afin de simplifier
les notations nous omettrons le conditionnement trivial et utiliserons simplement l'espérance.


