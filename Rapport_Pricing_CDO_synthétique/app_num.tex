\section{Application numérique}

\subsection{Tarification de tranches}

\subsubsection{Jeu de données}\label{subsubsec:data}

Pour notre étude nous étudierons $Q = 4$ tranches du CDO synthétique
CDX NA IG Series 19, il possède quatre différentes maturités (3, 5, 7
and 10 ans) et est basé sur un portefeuille de $n = 125$ contrats CDS.
Nous utiliserons les données fournies Table (\ref{tab:OkhrinXu}) par Okhrin et Xu (2017) pour la
maturité $T = 5$ ans sur 10 dates fournies entre 01/06/2014 et le 
15/03/2015.

\begin{table}[H]
\centering
\begin{tabular}{lccccc}
\toprule
Date & 0--3\% & 3--7\% & 7--15\% & 15--100\% & CDS \\
\midrule
2014/06/01 & 4.250 & 2.000 & 0.036 & 0.014 & 39 \\
2014/07/03 & 3.750 & 1.375 & 0.048 & 0.015 & 37 \\
2014/08/15 & 4.094 & 1.719 & 0.050 & 0.014 & 38 \\
2014/09/23 & 3.750 & 1.375 & 0.056 & 0.012 & 37 \\
2014/10/11 & 5.775 & 1.810 & 0.050 & 0.012 & 41 \\
2014/11/17 & 4.188 & 0.985 & 0.057 & 0.015 & 35 \\
2014/12/01 & 3.183 & 0.747 & 0.060 & 0.016 & 32 \\
2015/01/07 & 7.065 & 0.875 & 0.055 & 0.013 & 39 \\
2015/02/10 & 7.559 & 0.563 & 0.055 & 0.014 & 37 \\
2015/03/15 & 6.874 & 0.073 & 0.064 & 0.015 & 34 \\
\bottomrule
\end{tabular}
\caption{Spreads de 4 tranches du CDX NA IG Series 19 et les spreads des CDS associés}
\label{tab:OkhrinXu}
\end{table}

Le taux d'intérêt est pris constant à $r = 0.0014$ en accord avec la moyenne du LIBOR sur les
dates considérées et le taux de recouvrement est fixé à $R = 0.4$ en accord avec
la politique de l'entreprise Markit qui administre l'indice de ces produits. \cite{OhkrinXu}
Les taux de recouvrement moyens observés pour les obligations corporate senior unsecured sur 
longue période sont autour de 35--40 \%, selon les données historiques de Moody's Investors Service. Ces statistiques justifient l'usage conventionnel
d'un taux de recouvrement de 40 \% dans la calibration des modèles de CDS. \cite{Tresor}

\subsubsection{Copule gaussienne}

Dans cette partie nous allons exposer les premiers résultats obtenus pour un modèle simple de
copule gaussienne sous hypothèse d'un portefeuille homogène.

Les résultats de calibrations sur les données présentées en section (\ref{subsubsec:data}) sont
présentés Figure (\ref{fig:calibration_copule_gaussienne}) et montrent un résultat connu de la
littérature : le modèle gaussien présente des queues de distributions indépendantes et n'arrive
donc pas à simuler des évènement rares de défauts corrélés, les tranches equity sont bien approximées
mais les autres tranches sont surévaluées par rapport aux spreads observés.

\begin{figure}[H]
\centering
\includegraphics[width=1\textwidth]{images/spread_gaussian_1fact.png}
\caption{Comparaison pour chaque tranche des spreads observés et calculés par copule gaussienne à un facteur}
\label{fig:calibration_copule_gaussienne}
\end{figure}

Nous pouvons voir Figure (\ref{fig:corr_copule_gaussienne}) que la corrélation implicite du modèle gaussien
est très élevée oscillant autour de 0,7 pour les différentes dates.

\begin{figure}[H]
\centering
\includegraphics[width=0.8\textwidth]{images/spread_ev_gaussian_1fact.png}
\caption{Evolution du coefficient de corrélation implicite pour la copule gaussienne à un facteur}
\label{fig:corr_copule_gaussienne}
\end{figure}

\begin{figure}[H]
\centering
\includegraphics[width=0.8\textwidth]{images/smile_gaussian_1fact.png}
\caption{Correlation implicite par tranche pour la copule gaussienne à un facteur}
\label{fig:smile_gaussian_1fact}
\end{figure}